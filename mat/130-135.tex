% ======================================================================
%  IX. Całka oznaczona  –  strona 130 (1 : 1)
% ======================================================================
\documentclass[12pt,a4paper]{article}

\usepackage[utf8]{inputenc}
\usepackage[T1]{fontenc}
\usepackage[polish]{babel}
\usepackage{mathptmx}   % Times New Roman – tekst + wzory
\usepackage{amsmath,amssymb}
\usepackage{geometry}
\geometry{margin=2.5cm}
\usepackage{graphicx}   % jeśli jeszcze nie dodano

\pagestyle{plain}
\sloppy                       % zapobiega overfull boxes

\begin{document}
\setcounter{page}{130}

\begin{center}
{\bfseries IX. Całka oznaczona}
\end{center}

Na zakończenie, korzystając z własności wielomianów Legendre’a,
wyprowadzimy wzory redukcyjne wiążące trzy kolejne takie wielomiany.

Zauważamy na wstępie, że wielomian $x^{n}$ można wyrazić w postaci funkcji
liniowej jednorodnej zmiennych
$P_{0},P_{1},\dots ,P_{n}$ o stałych współczynnikach; jest to również
prawdą dla dowolnego wielomianu stopnia $n$.
Dlatego też
\[
xP_{n}=a_{0}P_{\,n+1}+a_{1}P_{n}+a_{2}P_{\,n-1}+a_{3}P_{\,n-2}+\dots,
\]
gdzie $a_{0},a_{1},a_{2},a_{3},\dots$ są współczynnikami stałymi.
Łatwo jest sprawdzić, że $a_{2}=a_{4}=a_{6}=\dots=0$.
Na przykład, aby obliczyć $a_{3}$, mnożymy obie strony tej równości
przez $P_{\,n-2}$ i całkujemy od $-1$ do $1$:
\[
\int_{-1}^{1} P_{n}\,x\,P_{\,n-2}\,dx
=
a_{0}\!\int_{-1}^{1} P_{\,n+1}P_{\,n-2}\,dx
+
a_{1}\!\int_{-1}^{1} P_{n}P_{\,n-2}\,dx
+
a_{3}\!\int_{-1}^{1} P_{\,n-2}^{2}\,dx+\dots
\]
W myśl (8) i (9) wszystkie całki z wyjątkiem jednej są zerami, stąd
\[
a_{3}\int_{-1}^{1} P_{\,n-2}^{2}\,dx = 0,\quad\text{skąd }a_{3}=0.
\]

Współczynnik $a_{1}$ jest także równy zeru, gdyż po lewej stronie
równości w ogóle nie występuje wyraz z $x^{n}$.
Aby wyznaczyć współczynnik $a_{0}$, porównujemy współczynniki przy
$x^{\,n+1}$ po obu stronach równości
\[
\frac{(2n-1)!!}{n!}=a_{0}\,\frac{(2n+1)!!}{(n+1)!},
\quad\text{skąd }a_{0}=\frac{\,n+1\,}{\,2n+1}.
\]
Wreszcie, aby wyznaczyć $a_{2}$, porównujemy wartości obu stron
równości dla $x=1$:
\[
1=a_{0}+a_{2},\quad\text{skąd }a_{2}=1-a_{0}=\frac{\,n\,}{\,2n+1}.
\]

Podstawiając znalezione współczynniki, otrzymujemy ostatecznie
\begin{equation}\label{eq:LegendreRec}
(n+1)P_{\,n+1}-(2n+1)\,x\,P_{n}+nP_{\,n-1}=0.\tag{11}
\end{equation}

Jest to właśnie szukany wzór redukcyjny, który umożliwia znajdowanie
kolejno wielomianów Legendre’a, wychodząc z $P_{0}=1,\;P_{1}=x$:
\[
P_{2}=\frac{3x^{2}-1}{2},\qquad
P_{3}=\frac{5x^{3}-3x}{2},\qquad
P_{4}=\frac{35x^{4}-30x^{2}+3}{8}.
\]

\bigskip
{\bfseries 321. Nierówności całkowe.}
W ustępach 133 i 144 wprowadziliśmy kilka nierówności dla sum.
Pokażemy teraz, w jaki sposób podobne nierówności można otrzymać dla całek.
Wszystkie występujące tu funkcje $p(x),q(x),v(x)$ są z założenia
całkowalne\textsuperscript{(1)}.

\begin{enumerate}
\item
W ustępie 133 mieliśmy nierówność (4), którą można napisać w postaci
\begin{equation}
\exp\frac{\displaystyle\sum p_{i}\ln a_{i}}{\displaystyle\sum p_{i}}
<
\frac{\displaystyle\sum p_{i}a_{i}}{\displaystyle\sum p_{i}}.
\tag{12}
\end{equation}
Rozpatrzmy teraz dwie funkcje $p(x)$ i $q(x)$ dodatnie,
określone w przedziale $\langle a,b\rangle$. Rozbijmy go punktami
\[
x_{0}=a<x_{1}<\dots<x_{i}<x_{i+1}<\dots<x_{s}=b.
\]
\end{enumerate}

\vfill
\noindent\textsuperscript{(1)} Z tego założenia wynika już całkowalność
i innych napotykanych tu funkcji; dla uzasadnienia wystarczy się powołać
na ustępy 299, II i 300, 4.

%--------------------------------------------------------------------

\newpage
\thispagestyle{plain}
\setcounter{page}{131}

\begin{center}
{\bfseries \S\;4. Niektóre zastosowania całek oznaczonych}
\end{center}

Przedział ten rozbijamy na podprzedziały o długościach
$\Delta x_{i}=x_{i+1}-x_{i}$.
Podstawmy teraz w nierówności~(12)
$p_{i}=p(x_{i})\Delta x_{i}$,
$a_{i}=\varphi(x_{i})$.
Otrzymujemy nierówność
\[
\exp\frac{\displaystyle\sum p(x_{i})\ln\varphi(x_{i})\,\Delta x_{i}}
         {\displaystyle\sum p(x_{i})\,\Delta x_{i}}
<
\frac{\displaystyle\sum p(x_{i})\varphi(x_{i})\,\Delta x_{i}}
     {\displaystyle\sum p(x_{i})\,\Delta x_{i}}.
\]

Wszystkie występujące tu sumy mają postać sum całkowych i przy
$\Delta x_{i}\!\to\!0$ dążą do odpowiednich całek.
W ten sposób w granicy otrzymujemy nierówność całkową
\[
\exp\frac{\displaystyle\int_{a}^{b} p(x)\ln\varphi(x)\,dx}
         {\displaystyle\int_{a}^{b} p(x)\,dx}
<
\frac{\displaystyle\int_{a}^{b} p(x)\varphi(x)\,dx}
     {\displaystyle\int_{a}^{b} p(x)\,dx},
\]
analogiczną z (12).
W szczególności dla $p(x)\equiv1$ mamy
\[
\exp\!\Bigl[\frac{1}{\,b-a\,}\int_{a}^{b}\ln\varphi(x)\,dx\Bigr]
<
\frac{1}{\,b-a\,}\int_{a}^{b}\varphi(x)\,dx.
\]

Wyrażenie stojące po prawej stronie nazywa się
\emph{średnią arytmetyczną} wartości funkcji $\varphi(x)$
w przedziale $\langle a,b\rangle$,  
a wyrażenie po lewej — \emph{średnią geometryczną} tej funkcji.

\medskip
{\bfseries 2)} Wyprowadzimy teraz nierówności całkowe analogiczne do
nierówności Cauchy’ego–Höldera oraz nierówności Minkowskiego [133, § 5, (7)]{}:
\[
\sum a_{i}b_{i}
<
\Bigl\{\sum a_{i}^{\,k}\Bigr\}^{1/k}\,
\Bigl\{\sum b_{i}^{\,k'}\Bigr\}^{1/k'},\tag{13}
\]

\[
\Bigl\{\sum(a_{i}+b_{i})^{k}\Bigr\}^{1/k}
<
\Bigl\{\sum a_{i}^{\,k}\Bigr\}^{1/k}
+
\Bigl\{\sum b_{i}^{\,k}\Bigr\}^{1/k},\tag{14}
\]

gdzie $k,k'>1$ oraz $\dfrac1k+\dfrac1{k'}=1$.

Niech dane będą w przedziale $\langle a,b\rangle$ dwie funkcje dodatnie
$p(x)$ i $\psi(x)$;
rozbijmy, jak wyżej, ten odcinek punktami $x_{i}$ i podstawmy w (13)
\[
a_{i}=\varphi(x_{i})\,\Delta x_{i}^{1/k},
\qquad
b_{i}=\psi(x_{i})\,\Delta x_{i}^{1/k'},
\]
a w (14)
\[
a_{i}=\varphi(x_{i})\,\Delta x_{i}^{1/k},
\qquad
b_{i}=\psi(x_{i})\,\Delta x_{i}^{1/k}.
\]

Otrzymujemy
\[
\sum \varphi(x_{i})\psi(x_{i})\,\Delta x_{i}
<
\Bigl\{\sum[\varphi(x_{i})\,\Delta x_{i}]^{k}\Bigr\}^{1/k}
\Bigl\{\sum[\psi(x_{i})\,\Delta x_{i}]^{k'}\Bigr\}^{1/k'},
\]
\[
\Bigl\{\sum[\varphi(x_{i})+\psi(x_{i})]^{k}\,\Delta x_{i}\Bigr\}^{1/k}
<
\Bigl\{\sum[\varphi(x_{i})\,\Delta x_{i}]^{k}\Bigr\}^{1/k}
+
\Bigl\{\sum[\psi(x_{i})\,\Delta x_{i}]^{k}\Bigr\}^{1/k}.
\]

Przechodząc do granicy przy $\Delta x_{i}\to0$, otrzymujemy ostatecznie
\[
\int_{a}^{b} \varphi\psi dx
<
\Bigl\{\int_{a}^{b} \varphi^{k}\,dx\Bigr\}^{1/k}
\Bigl\{\int_{a}^{b} \psi^{k'}\,dx\Bigr\}^{1/k'},
\tag{13$^{\ast}$}
\]
oraz
\[
\Bigl\{\int_{a}^{b}[\varphi+\psi]^{k}\,dx\Bigr\}^{1/k}
<
\Bigl\{\int_{a}^{b} \varphi^{k}\,dx\Bigr\}^{1/k}
+
\Bigl\{\int_{a}^{b} \psi^{k}\,dx\Bigr\}^{1/k}.
\tag{14$^{\ast}$}
\]
%--------------------------------------------------------------------

\newpage
\thispagestyle{plain}
\setcounter{page}{132}

\begin{center}
{\bfseries IX. Całka oznaczona}
\end{center}

Zwracamy uwagę na szczególne przypadki tych wzorów, kiedy
$k=k'=2$:

\[
(\text{13'})\qquad
\int_{a}^{b}\varphi\psi\,dx
\le
\sqrt{\int_{a}^{b}\varphi^{2}\,dx}\;
\sqrt{\int_{a}^{b}\psi^{2}\,dx},
\]

i

\[
(\text{14'})\qquad
\sqrt{\int_{a}^{b}[\varphi+\psi]^{2}\,dx}
\le
\sqrt{\int_{a}^{b}\varphi^{2}\,dx}
+
\sqrt{\int_{a}^{b}\psi^{2}\,dx}.
\]

Pierwsza z nich jest nierównością Buniakowskiego.
Drugą łatwo otrzymuje się z niej przez podniesienie stronami do kwadratu.

\bigskip
{\bfseries 2)} Rozpatrzmy jeszcze nierówność Jensena [144 (12$^{\prime}$)]:
\[
(\text{15})\qquad
f\!\Bigl(\frac{\sum p_{i}x_{i}}{\sum p_{i}}\Bigr)
\le
\frac{\sum p_{i}f(x_{i})}{\sum p_{i}},
\]
o funkcji $f(x)$ zakładamy tu, że jest ona wypukła w pewnym przedziale $X$,
w którym leżą punkty $x_{i}$, a $p_{i}$ są liczbami dodatnimi.
Przypuśćmy, że funkcja $\varphi(x)$ jest określona w pewnym przedziale
$\langle a,b\rangle$ i jej wartości leżą w przedziale $X$,
a funkcja $p(x)$ jest dodatnia i również określona w przedziale
$\langle a,b\rangle$.
Niech teraz $x_{i}$ oznaczają punkty podziału tego odcinka.
Dawne $x_{i}$ we wzorze (15) zastąpimy przez $\varphi(x_{i})$,
a $p_{i}$ przez $p(x_{i})\Delta x_{i}$.
Przechodząc następnie, tak jak wyżej, od sum całkowych do całek,
otrzymamy \emph{całkową nierówność Jensena}:
\[
f\!\left(
      \frac{\displaystyle\int_{a}^{b} p(x)\,\varphi(x)\,dx}
           {\displaystyle\int_{a}^{b} p(x)\,dx}
   \right)
<
\frac{\displaystyle\int_{a}^{b} p(x)\,f\!\bigl(\varphi(x)\bigr)\,dx}
     {\displaystyle\int_{a}^{b} p(x)\,dx}\, .
\]


%--------------------------------------------------------------------
\section*{\S\;5. Przybliżone obliczanie całek oznaczonych}

\subsection*{322. Postawienie zadania. Metoda prostokątów i metoda trapezów}

Przypuśćmy, że mamy znaleźć całkę
$\displaystyle\int_{a}^{b} f(x)\,dx$,
gdzie $f(x)$ jest pewną funkcją ciągłą określoną w przedziale
$\langle a,b\rangle$.
W § 3 podaliśmy wiele przykładów obliczania takich całek
albo za pomocą funkcji pierwotnej, jeśli można ją było przedstawić
w postaci skończonej, albo też — omijając funkcję pierwotną —
za pomocą różnych chwytów, najczęściej dość sztucznych.
Zauważmy jednak, że wszystkie te metody dadzą się zastosować
jedynie do dość wąskiej klasy całek; poza tą klasą musimy się uciekać
do metod rachunku przybliżonego.

Obecnie poznamy najprostsze z metod, w których wzory przybliżone
na całkę wykorzystują pewną liczbę wartości funkcji podcałkowej,
obliczanych dla pewnych wartości (zazwyczaj równoodległych) zmiennej
niezależnej.

Pierwsze spośród takich wzorów otrzymuje się najprościej z rozważań
geometrycznych. Traktując całkę oznaczoną
$\displaystyle\int_{a}^{b} f(x)\,dx$
jako pole pewnej figury geometrycznej ograniczonej krzywą $y=f(x)$
[294], postawimy sobie za zadanie znalezienie tego pola.

%--------------------------------------------------------------------

\newpage
\thispagestyle{plain}
\setcounter{page}{133}

Przede wszystkim zastosujemy powtórnie tę samą myśl, która doprowadziła nas
do pojęcia całki oznaczonej.
Rozbijemy mianowicie całą figurę (rys.\,6) na paski równej szerokości
$\Delta x_{1}=\dfrac{b-a}{n}$\,, a następnie każdy pasek zastępujemy w przybliżeniu przez prostokąt
o wysokości równej jednej z rzędnych. Prowadzi to do wzoru
\[
\int_{a}^{b}f(x)\,dx
= \frac{b-a}{n}\,
  \bigl[f(\xi_{0})+f(\xi_{1})+\dots+f(\xi_{\,n-1})\bigr],
\]
gdzie $x_{i}<\xi_{i}<x_{i+1}$ ($i=0,1,\dots ,n-1$).
Pole krzywoliniowej figury zastępujemy tu polem pewnej figury schodkowej
złożonej z prostokątów (innymi słowy, całkę oznaczoną zastępujemy sumą
całkową). Taki przybliżony sposób obliczania całki nazywa się
\emph{metodą prostokątów.}

\begin{figure}[h]            % [h] – umieść rysunki w tym miejscu
  \centering
  %--- Rysunek 6 --------------------------------------------------------
  \begin{minipage}[b]{0.45\textwidth}
    \centering
    \includegraphics[width=\linewidth]{Rys6.png}
    \caption{}
  \end{minipage}
  \hfill
  %--- Rysunek 7 --------------------------------------------------------
  \begin{minipage}[b]{0.45\textwidth}
    \centering
    \includegraphics[width=\linewidth]{Rys7.png}
    \caption{}
  \end{minipage}
\end{figure}


W praktyce bierze się zazwyczaj
\[
\xi_{i}=\frac{x_{i}+x_{i+1}}{2}=x_{\,i+\tfrac12}.
\]
Jeśli odpowiednią rzędną $f(\xi_{i})=f(x_{\,i+\tfrac12})$
oznaczymy przez $y_{\,i+\tfrac12}$, to ostatni wzór przyjmie postać
\[
(\text{1})\qquad
\int_{a}^{b}f(x)\,dx =
\frac{b-a}{n}\bigl(y_{\,\tfrac12}+y_{\,\tfrac32}+\dots+y_{\,n-\tfrac12}\bigr).
\]

W dalszym ciągu mówiąc o metodzie prostokątów będziemy mieli zawsze na myśli
ten właśnie wzór.

Interpretacja geometryczna prowadzi również do innego wzoru przybliżonego,
który jest często stosowany.
Zastępujemy mianowicie daną krzywą wpisaną w nią łamaną o wierzchołkach
w punktach $(x_{i},y_{i})$, gdzie $y_{i}=f(x_{i})$ ($i=0,1,\dots ,n-1$).
W ten sposób zastępujemy naszą figurę krzywoliniową przez inną,
składającą się z pewnej liczby trapezów (rys.\,7).
Jeśli — jak poprzednio — podzielimy przedział $\langle a,b\rangle$
na jednakowe części, to pola tych trapezów będą odpowiednio równe
\[
\frac{b-a}{n}\,\cdot\,\frac{y_{0}+y_{1}}{2},\quad
\frac{b-a}{n}\,\cdot\,\frac{y_{1}+y_{2}}{2},\;\dots,\;
\frac{b-a}{n}\,\cdot\,\frac{y_{\,n-1}+y_{n}}{2}.
\]
Dodając te pola otrzymamy nowy wzór przybliżony
\[
(\text{2})\qquad
\int_{a}^{b}f(x)\,dx =
\frac{b-a}{n}\Bigl(
\frac{y_{0}+y_{n}}{2}+y_{1}+y_{2}+\dots+y_{\,n+1}\Bigr).
\]

Jest to tak zwany \emph{wzór trapezów.}

%--------------------------------------------------------------------

\newpage
\thispagestyle{plain}
\setcounter{page}{134}
Można udowodnić, że przy przejściu $n\to\infty$
błąd w obu wyprowadzonych wzorach dąży do zera.
Wobec tego oba te wzory przy dostatecznie dużym $n$
dają przybliżoną wartość szukanej całki z dowolnie dużą dokładnością.

Jako przykład weźmy znaną nam już całkę
\[
\int_{0}^{1}\frac{dx}{1+x^{2}}=\frac{\pi}{4}=0{,}785398\dots
\]
i obliczymy jej wartość przybliżoną za pomocą obu tych wzorów,
biorąc $n=10$ i uwzględniając 4 miejsca dziesiętne.

%––– WEDŁUG METODY PROSTOKĄTÓW –––––––––––––––––––––––––––––––––––––
\paragraph*{Według metody prostokątów mamy:}

\[
\begin{array}{@{}r@{\;=\;}l@{\qquad}r@{\;=\;}l@{\qquad}
              r@{\;=\;}l@{\qquad}r@{\;=\;}l@{}}
x_{\frac12}  & 0{,}05 & y_{\frac12}  & 0{,}9975 &
x_{\frac{11}2} & 0{,}55 & y_{\frac{11}2} & 0{,}7678 \\[2pt]
x_{\frac32}  & 0{,}15 & y_{\frac32}  & 0{,}9780 &
x_{\frac{13}2} & 0{,}65 & y_{\frac{13}2} & 0{,}7030 \\[2pt]
x_{\frac52}  & 0{,}25 & y_{\frac52}  & 0{,}9412 &
x_{\frac{15}2} & 0{,}75 & y_{\frac{15}2} & 0{,}6400 \\[2pt]
x_{\frac72}  & 0{,}35 & y_{\frac72}  & 0{,}8909 &
x_{\frac{17}2} & 0{,}85 & y_{\frac{17}2} & 0{,}5866 \\[2pt]
x_{\frac92}  & 0{,}45 & y_{\frac92}  & 0{,}8316 &
x_{\frac{19}2} & 0{,}95 & y_{\frac{19}2} & 0{,}5256
\end{array}
\]

\[
\underline{\text{Suma }\,7{,}8562}
\quad\Longrightarrow\quad
\frac{\displaystyle 7{,}8562}{10}=0{,}78562
\]

%–––– ZE WZORÓW TRAPEZÓW OTRZYMUJEMY ––––––––––––––––––––––––––––––––
\paragraph*{Ze wzorów trapezów otrzymujemy:}

\begin{center}
%–– lewy blok ––
\begin{minipage}[t]{0.45\textwidth}
\centering
\begin{tabular}{@{}r@{\,=\,}l\quad r@{\,=\,}l@{}}
$x_{0}$  & 0{,}0 & $y_{0}$  & 1{,}0000\\
$x_{10}$ & 1{,}0 & $y_{10}$ & 0{,}5000\\ \hline
\end{tabular}

\vspace{2pt}
Suma \; 1,5000
\end{minipage}
\hfill
%–– prawy blok ––
\begin{minipage}[t]{0.45\textwidth}
\centering
\begin{tabular}{@{}r@{\,=\,}l\quad r@{\,=\,}l@{}}
$x_{1}$ & 0{,}1 & $y_{1}$ & 0{,}9901\\
$x_{2}$ & 0{,}2 & $y_{2}$ & 0{,}9615\\
$x_{3}$ & 0{,}3 & $y_{3}$ & 0{,}9174\\
$x_{4}$ & 0{,}4 & $y_{4}$ & 0{,}8621\\
$x_{5}$ & 0{,}5 & $y_{5}$ & 0{,}8000\\
$x_{6}$ & 0{,}6 & $y_{6}$ & 0{,}7353\\
$x_{7}$ & 0{,}7 & $y_{7}$ & 0{,}6711\\
$x_{8}$ & 0{,}8 & $y_{8}$ & 0{,}6098\\
$x_{9}$ & 0{,}9 & $y_{9}$ & 0{,}5525\\ \hline
\end{tabular}

\vspace{2pt}
Suma \; 7,0998
\end{minipage}
\end{center}

\[
\frac{1}{10}\!\Bigl(\,\frac{1{,}5000}{2}+7{,}0998\Bigr)=0{,}78498
\]

Oba otrzymane wyniki przybliżone mają mniej więcej ten sam rząd
dokładności — różnią się one od wartości dokładnej o mniej niż
$0{,}0005$ (w jedną i drugą stronę).

Czytelnik zdaje sobie oczywiście sprawę z tego, że w tym przypadku
moglibyśmy oszacować błąd tylko dlatego, że od początku znamy dokładną
wartość tej całki. Na to, żeby oba wzory były rzeczywiście przydatne do
obliczeń przybliżonych, potrzebne są wygodne wzory do obliczania błędu,
które pozwalałyby nie tylko oszacować błąd, ale również dobierać $n$ w
ten sposób, by zagwarantować żądaną dokładność. Do tego zagadnienia
wrócimy jeszcze w ustępie 325.

%--------------------------------------------------------------------

\newpage
\thispagestyle{plain}
\setcounter{page}{135}
\subsection*{323. Interpolacja paraboliczna}

Aby obliczyć przybliżoną wartość całki
$\displaystyle\int_{a}^{b}f(x)\,dx$
można spróbować zastąpić funkcję $f(x)$ aproksymującym ją wielomianem
\begin{equation}
(3)\qquad
y=P_{k}(x)=a_{0}x^{k}+a_{1}x^{\,k-1}+\dots+a_{k-1}x+a_{k}
\end{equation}
i przyjąć
\[
\int_{a}^{b}f(x)\,dx\;\approx\;\int_{a}^{b}P_{k}(x)\,dx .
\]

Inaczej mówiąc, zastępujemy tu, przy obliczaniu pola, daną „krzywą”
$y=f(x)$ przez parabolę (3) stopnia $k$. W związku z tym metoda ta
otrzymała nazwę \emph{interpolacji parabolicznej.}

Wielomian interpolacyjny $P_{k}(x)$ wybiera się najczęściej w sposób
następujący. W przedziale $\langle a,b\rangle$ bierzemy
$k+1$ wartości zmiennej niezależnej
$\xi_{0},\xi_{1},\dots,\xi_{k}$ i dobieramy wielomian $P_{k}(x)$ tak,
ażeby dla wybranych wartości zmiennej niezależnej przyjmował on te same
wartości, co funkcja $f(x)$. Wielomian spełniający taki warunek jest, jak wiemy [128], jednoznacznie określony i można go wyrazić \emph{wzorem interpolacyjnym Lagrange’a}:
\[
\begin{aligned}
P_{k}(x)=&
\frac{(x-\xi_{1})(x-\xi_{2})\dots(x-\xi_{k})}
     {(\xi_{0}-\xi_{1})(\xi_{0}-\xi_{2})\dots(\xi_{0}-\xi_{k})}\,
     f(\xi_{0})
\\[4pt]
&+\,\frac{(x-\xi_{0})(x-\xi_{2})\dots(x-\xi_{k})}
        {(\xi_{1}-\xi_{0})(\xi_{1}-\xi_{2})\dots(\xi_{1}-\xi_{k})}\,
        f(\xi_{1})
+\;\dots\\[4pt]
&+\,\frac{(x-\xi_{0})(x-\xi_{1})\dots(x-\xi_{k-1})}
        {(\xi_{k}-\xi_{0})(\xi_{k}-\xi_{1})\dots(\xi_{k}-\xi_{k-1})}\,
        f(\xi_{k}) .
\end{aligned}
\]


Do scałkowania otrzymuje się w ten sposób wyrażenie liniowe względem
$f(\xi_{0}),\dots ,f(\xi_{k})$, którego współczynniki nie zależą już od tych wartości.
Współczynniki te można wyznaczyć raz na zawsze i posługiwać się nimi przy
obliczaniu przybliżonym całki dowolnej funkcji $f(x)$ w danym przedziale
$\langle a,b\rangle$.

W najprostszym przypadku dla $k=0$ funkcję $f(x)$ zastępujemy
po prostu stałą $f(\xi_{0})$, gdzie $\xi_{0}$ jest dowolnym punktem
z przedziału $\langle a,b\rangle$, na przykład jego środkiem
$\xi_{0}=(a+b)/2$. Wtedy w przybliżeniu mamy
\[
\makebox[0pt][l]{\hspace*{-2em}(4)\quad}   % numer w lewym marginesie
\int_{a}^{b} f(x)\,dx \;\approx\; (b-a)\,
      f\!\left(\frac{a+b}{2}\right).
\]


Interpretując geometrycznie ten wzór widzimy, że pole figury
krzywoliniowej jest tu zastąpione polem prostokąta o wysokości równej
rzędnej funkcji $f(x)$ w środku przedziału.

Przy $k=1$ funkcja $f(x)$ zostaje zastąpiona przez funkcję liniową
$P_{1}(x)$, która dla $x=\xi_{0}$ i $x=\xi_{1}$ ma takie same wartości
jak $f(x)$. Jeśli przyjąć $\xi_{0}=a,\;\xi_{1}=b$, to będzie
\[
P_{1}(x)=\frac{x-b}{a-b}f(a)+\frac{x-a}{b-a}f(b)
\]
i jak łatwo sprawdzić
\[
\int_{a}^{b} P_{1}(x)\,dx=(b-a)\,\frac{f(a)+f(b)}{2}.
\]
W ten sposób otrzymaliśmy w przybliżeniu
\[
(\text{5})\qquad
\int_{a}^{b}f(x)\,dx \approx (b-a)\,\frac{f(a)+f(b)}{2}.
\]

\end{document}
